\documentclass[asi]{picINSA}
\usepackage{cite}

\begin{document}

    \titreGeneral{Bibliographie}
	\sousTitreGeneral{\newline Sujet RPG 1}
	\titreAcronyme{\LARGE IHME}
	\version{1.00}
	\referenceVersion{bibliographie}
	
	\couverture{}

\tableofcontents{}

\chapter{Génération de l'intrigue}

\chapter{Création des personnages}
~\\
Il y a plusieurs approches possibles pour rendre les personnages intéressant dans la narration interactive. Nous allons d'abord voir comment sont représentées les actions des personnages non joueurs, et comment un joueur peut agir dessus. Nous verrons ensuite comment modéliser la personnalité des personnages, pour agir automatiquement sur le choix des actions d'un personnage.\\

\section{Interactions humain-personnage}

Dans ~\cite{IRIS:conf/aamas/CavazzaCM2002}, une histoire se déroule sous les yeux d'un joueur humain. Celui-ci peut à tout moment décider d'intervenir, en agissant dans la scène (ex : déplacement d'un objet), ou bien en donnant des conseils ou des indications à un personnage. Leur système utilise une reconnaissance vocale pour interprêter ce que le joueur veut dire au personnage. Ce message est ensuite transformé pour être comprise par le personnage, qui va changer son comportement en fonction. Par exemple, si le personnage cherche un objet mais ne le trouve pas dans la pièce où il se trouve, le joueur peut donner une indication au personnage, qui va aller au bon endroit.\\

Les actions du personnages sont représentées par un HTN : Hierarchical Task Networks. [figure]\\

La racine de cet arbre contient l'objectif principal du personnage. Les noeuds en dessous sont les différents sous-objectifs nécessaires à court terme pour réaliser l'objectif principal. Pour chaque sous-objectif, une liste d'actions possible à effectuer pour le réaliser donne au personnage différents moyens d'arriver à ses fins. Le choix dans ces actions s'effectue d'abord simplement en prenant pour chaque sous-objectif les actions de gauche à droite, puis lorsque cette action est impossible, prends la suivante. Un poids peut être associé à chacune de ces actions pour éviter un choix qui pourrait être mal vu par les autres personnages par exemple.\\

\section{Modélisation de la personnalité}

Dans un jeu vidéo, beaucoup d'élements peuvent être scripté pour faire réagir les personnages d'une certaine façon. En revanche, en laissant le joueur assez libre dans ses actions, il faut trouver un autre moyen pour que les personnage non joueurs (PNJ) réagissent automatiquement, que ce soit par leurs actions directes, ou les émotions qu'ils laissent paraîtres.\\

Comme présenté dans ~\cite{IRIS:conf/aiide/OchsSC2008}, la plupart des jeux de rôles ou des jeux d'aventures plongent le joueurs dans un rôle précis, avec un contexte déjà établie. L'impact des actions du joueurs sont alors généralement simplifiés a :

\begin{itemize}
\item un alignement : bien ou mal, loyal ou chaotic, pouvant limiter
  les actions possibles du joueur ou générer différentes réactions de
  la part des PNJs

\item une réputation, connu de tous les PNJs, qui peut modifier
  l'attitude des PNJs envers le joueur (par exemple, un joueur avec
  une mauvaise réputation pourra se faire attaquer dès qu'il croisera
  un garde).
\end{itemize}

Des résultats tirés d'études psychologique peuvent être utilisés afin de représenter plus précisément les relations sociales
  
\bibliography{sources/bibliographieST}{}
\bibliographystyle{plain}
\end{document}
