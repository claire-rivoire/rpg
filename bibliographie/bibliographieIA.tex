\documentclass[asi]{picINSA}

\begin{document}

    \titreGeneral{Bibliographie}
	\sousTitreGeneral{\newline Sujet RPG 1}
	\titreAcronyme{\LARGE IHME}
	\version{1.00}
	\referenceVersion{bibliographie}
	
	\couverture{}

\tableofcontents{}

\chapter{Introduction}

\chapter{IA décisionnelles (Intelligence fixe)}
\section{Arbre de decision}
Les intelligences artificielles de jeux vidéo les plus basiques possèdent simplement leur comportement décrit en dur dans le code, sous forme d’un arbre de décision. \\
%{\begin{figure} [ht]
%         \begin{center}
%        \includegraphics[scale=0.3]{./images/arbre.pdf}
%         \end{center}
%\end{figure}

Mais cette méthode possède des limites : lorsque le nombre d’actions possibles et le nombre de stimulations de l’environnement augmente, c’est la complexité de l’IA qui augmente également car il faut envisager toutes les situations possibles.

\section{Machines à état fini}
\section{Fuzzy Logic}
  
\chapter{IA Evolutives (Machine Learning)}
\section{Réseau de neurones}

\section{Algorithmes génétiques}
\section{Apprentissage supervisé et non-supervisé}
Les algorithmes d’intelligence artificielle fonctionnant par apprentissage sont nombreux. Dans cette catégorie peuvent rentrer les réseaux de neurones, les algorithmes génétiques . . . Dans tous les cas, deux méthodes d’apprentissages sont possibles : l’apprentissage supervisé et l’apprentissage non supervisé.

\subsection{Apprentissage supervisé}
Cette méthode demande une base d’exemple sur laquelle fonder l’apprentissage, en l’occurrence, l’enregistrement d’un joueur humain jouant au jeu. L’IA analyse la situation dans laquelle se trouve le joueur et enregistre ses actions, pour les reproduire lorsqu’elle se retrouve dans la même situation. Cette méthode est assez contraignante pour les développeurs qui doivent passer un certain temps à jouer pour rendre l’IA efficace.

\subsection{Apprentissage non-supervisé}
Cela consiste à créer des intelligences artificielles qui s’adaptent toutes seules à leur environnement. Cette méthode a l’avantage d’être assez légère pour le développeur qui doit juste créer un système de "récompense" quand l’IA agit correctement, et inversement  quand elle fait des erreurs. Mais elle est plus difficile à réaliser, et plus souvent sujette à des erreurs non prévues. Si on prévoit mal la réaction de l’IA par rapport à ses stimulations extérieures, on risque de rentrer dans un comportement illogique.
\section{Algorithmes de Vie Artificielle}
  
\chapter{Systèmes multi-agents}

\chapter{Modèlisation de la parsonnalité et des relation sociales}
Lorsque l'on veut intégrer dans un jeu vidéo des PNJ (personnages non jouables) assez développés, intervient un moment ou l'on se heurte au problème de la personnalité du personnage. En effet si on veut que les différents PNJ n'aient pas les mêmes réactions face à ce qui leur arrive, il faut pouvoir leur donner une personnalité. Par exemple, un villageois et un chevalier ne réagiront pas de la même façon face à une attaque : on peut considérer que cela leur vient de la différence dans leur niveau de courage. De même, si on veut mettre en place une relation perticulière vis-à-vis du joueur ou des autres personnages, il faut pouvoir représenter les émotions des personnages, et que celles-ci influent sur leurs réactions sociales. Par exemple, un personnage sera moins enclin à donner des informations au joueur si celui-ci l'a attaqué que s’il l'a aidé. 

\section{Modèles de personnalité}
Différentes approches existent pour représenter la personnalité d'un personnage :
\begin{itemize}
\item L'approche par catégorie : on spécifie des comportements en fonction d'une catégorie comme par exemple « gentil/méchant » ou « peureux/courageux » comme dans le cas du villageois et du chevalier. Les différentes catégories sont spécifiées à l'avance dans le code du jeu et n'évoluent pas.
\item L'approche qui se base sur des traits de caractères. Par exemple, une capacité de « courage » peut être utilisée pour savoir si le personnage va répondre à une attaque en fuyant ou en se battant. Cette méthode permet par la suite d'implémenter une modification des attributs en cours de jeu, en fonction de ce qui arrive au personnage.  
\end{itemize}
Les représentations de personnalités sont souvent basées sur des modèles proposés par des grands noms de la psychologie  :
\begin{itemize}
\item Le modèle de personnalité de Hans Eysenck, qui considère 2 dimensions: Extroversion/ Introversion et  le Névrosisme/Stabilité émotionnelle
\item Le modèle OCEAN qui tire son nom de ses 5 dimensions : Ouverture à l'expérience, Conscienciosité, Extraversion, Agréabilité, et Névrosisme
\item Le MBTI (Myers Briggs Type Indicator) qui fonctionne avec 4 dimensions : Extraversion/Introversion, Sensation/Intuition, Pensée/Sentiments, Perception/Jugement
\end{itemize}
L'utilisation de ces modèles pour concevoir des IA de jeu vidéo permet d'améliorer leur crédibilité pour le joueur.

\section{Relations sociales}
Dans un jeu vidéo, les PNJ sont généralement là pour interagir avec le joueur, et, si on veut que leur comportement soit vraisemblable, il faut mettre en place une représentation des relations sociales entre les personnages. \\

Les relations sont généralement représentées, chez chaque personnage par un nombre fini de variables, qui s’appliquent à un autre personnage (le joueur ou bien un autre PNJ) avec qui il est déjà entré en communication. 4 variables sont généralement utilisées :
\begin{itemize}
\item Le degré d'appréciation que le personnage a pour un autre.
\item la dominance, c'est-à-dire le pouvoir que le personnage pense pouvoir exercer sur l'autre (et si elle est négative, c'est le pouvoir qu'il pense que l'autre peut exercer sur lui).
\item la solidarité, qui correspond au degré de similarité entre les contextes des 2 personnages, par exemple, l'âge, le sexe, les valeurs, les croyances ...
\item la familiarité, qui va caractériser la tendance du personnage à échanger des informations avec un autre.
\end{itemize}
Ces variables sont spécifiques à chacun des 2 personnages qui sont en relation et peuvent évoluer symétriquement ou non. Leur évolution se fait en fonction des interactions, par exemple, plus deux personnages parlent ensemble, plus leur niveau de familiarité augmente. Et réciproquement, ces variables influent sur les interactions possibles entre les personnages, par exemple, leur niveau de familiarité doit atteindre un certain seuil avant qu'ils ne puisse partager des informations confidentielles. 


\chapter{Conclusion}

  
\end{document}
