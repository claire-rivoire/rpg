\documentclass[asi]{picINSAIA}
\usepackage{cite}

% Bout de code pour enlever le ``chapitre'' écrit à chaque fois
\makeatletter
\def\@makechapterhead#1{
  {\parindent \z@ \raggedright \normalfont
    \interlinepenalty\@M
    \Huge\bfseries  \thechapter.\quad #1\par\nobreak
    \vskip 40\p@
  }}
\makeatother
% Enlève les doubles pages blanches inutiles
\let\cleardoublepage\clearpage

\begin{document}

    \titreGeneral{Bibliographie}
	\sousTitreGeneral{\newline Sujet RPG 1 : Intelligences Artificielles}
	\titreAcronyme{\LARGE IHME}
	\version{1.00}
	\referenceVersion{bibliographie}
	
	\couverture{}
	
\newpage\null\thispagestyle{empty}\newpage
\tableofcontents{}

\chapter{Introduction}
De nos jours, les jeux vidéos prennent de plus en plus d'ampleur, et les joueurs sont devenus plus exigeants que jamais. 
Si la qualité graphique des jeux a souvent été mise en avant pour les vendre, celle-ci évolue de moins en moins, laissant plus de place à l'évolution des Intelligence Artificielle (IA) et du gameplay.

La qualité d'un jeu vidéo dépend de beaucoup de facteurs, et notament, du sentiment d'immersion resenti par le joueur. Ce sentiment d'immersion est optimal, lorsque le monde dans lequel il est plongé est crédible, et cette crédibilité repose souvent sur le comportement des Personnages Non Jouables (PNJ). Ceux-ci ne peuvent plus se contenter d'un comportement basique et sans réfléxion derrière, ou sinon le joueur se rendra vite compte qu'il tourne en rond dans un univers fermé. Pour éviter cela, l'IA du PNJ est primordiale. \\
De plus, avec l'augmentation de la puissance de calcul disponible pour le grand publi, les developpeurs disposent de plus d'opportunités de développement d'IA plus puissantes et plus intelligentes.\\

Il existe différentes façons de doter les PNJ d'une IA. Dans cette synthèse bibliographique, nous allons commencer par présenter les IA de type décisionnelles, qui possèdent un fonctionnement fixé et non évolutif. Ensuite, nous parlerons des IA évolutives, capables d'apprendre et de s'améliorer au cours du jeu, ainsi que des systèmes multi-agents. Pour finir nous évoquerons l'implémentation de la personnalité des IA, qui permet de les doter d'un comportement plus riche et plus vraisemblable.


\chapter{IA décisionnelles (Intelligence fixe)}
\section{Arbre de decision et machines à états finis}
Les intelligences artificielles de jeux vidéo les plus basiques possèdent simplement leur comportement décrit en dur dans le code, sous forme d’un arbre de décision. \\
Les machines a états finis sont une version améliorée de ces arbres de décisions, dont la conception est plus réfléchi et permet un minimum de réfléxion de l'IA face à la situation dans laquelle elle est. Un exemple très simple serait celui d'un garde ~\cite{MindGames}. Aussitôt que le garde voit le jouer et possède un certain montant de point de vie, il va attqué celui-ci. Si le garde passe sous un certain montant de vie, il fuit, et si le joueur fuit, il retourne à son poste. Ce cas peut évidemment être conplexifié, mais représente bien les machines à états finis. On peut retrouver d'autres exmples de telles machines dans ~\cite{khoo2002applying}.\\

L'avantage de ces techniques, est leur facilité de mise place, et un temps de calcul très court, mais elles possèdent des limites : lorsque le nombre d’actions possibles et le nombre de stimulations de l’environnement augmentent, c’est la complexité de l’IA qui augmente également car il faut envisager toutes les situations possibles. En terme de temps de développement cela devient très complexe et peut provoquer des actions illogiques au regard du joueur.

\section{Fuzzy Logic}
Le problème avec les machines à état fini est que les joueurs peuvent, à force de jouer, reconnaitre un pattern d'actions de la part de l'IA et alors leur intérêt pour le jeu sera moindre. Afin de préserver une part de ``surprise'' de la part de l'IA envers le joueur, certains développeurs intègrent ce que l'on appelle une logique floutée (fuzzy logic).

L'intelligence sera la même que précédemment, mais celle-ci sera pondérée par d'autres paramètres. Le but est d'avoir des décisions non-binaires, c'est-à-dire ne semblant pas être de simples SI ... ALORS dans des machines à états finis ~\cite{CompGamesWithIntel}.
Ces paramètres peuvent être de simple poids aléatoires sur les données permettant la prise de décision, mais sont généralement des poids basés sur la personnalité que les développeurs souhaitent donner à leur IA. Cela peut permettre des IA plus ou moins aggressives et plus ou moins prévisibles en fonction des poids.\\

Par exemple, le FPS Unreal possède une IA basée sur cette logique, et permet aux PNJ, en fonction des évènements de la bataille, de fuir, se cacher, créer des embuscasdes, etc. ~\cite{CompGamesWithIntel}.
De même le jeu de statégie Civilisation: Call to Power, une seule IA est développée pour le jeu. Celle-ci est ensuite pondérée afin de la faire varier pour représenter un comportement différent pour chaque différente civilisation ~\cite{CompGamesWithIntel}. 


\chapter{IA Evolutives (Machine Learning)}
Le problème des IA décisionnelles est leur manque d'évolution. Elle ne sont pas adaptatives, et c'est pourquoi d'autres techniques plus élaborées ont vu le jour. Le but de celles-ci est de s'adapter au joueur et de lui permettre d'avoir un challenge tout au long du jeu, avec des IA évolutives. Nous allons ici présenter quelques types d'IA évolutives.

\section{Réseaux de neurones et algorithmes génétiques}
Les réseaux de neurones est un technique que permet de faire évoluer l'IA avec le joueur. Lorsque celui-ci joue, il va conduire l'IA a prendre diverse décisions, et amener à un résultat, qui sera ensuite utilisé pour améliorer le réseau de neurones. Le même raisonnement est appliqué pour les algorithmes génétiques, et permettent de faire évoluer l'IA. Dans les deux cas, on est face à un problème d'optimisation de l'IA, mais c'est l'implémentation qui diffère ~\cite{alvarez2013machine}.

Théoriquement, ces IA pourraient évoluer en permanence, jusqu'à un état d'excellence dans leurs décisions, mais pourraient aussi faire l'inverse, et arriver à un état où leurs décisions deviendraient illogiques ~\cite{CompGamesWithIntel}.\\
C'est pour cela que ces techniquse sont peu utilisées. Et outre le fait qu'elles sont difficiles à mettre en place, si une IA prend des décisions illogiques, alors ce n'est pas une bonne IA, et donc peu de développeurs prennent ce risque.\\

\section{Algorithmes de Vie Artificielle}
Une IA de Vie Artificielle est une IA composée de ``sous-IA''. Ces sous-IA simples fonctionnent comme des IA normales et prennent des décisons qui permettent à l'IA supérieur de prendre une décison, considérant les décisions des autres, et amène ainsi à créer une IA plus complexe.
Ce phénomène est appelé l'``énergence'' et l'exemple représentant le mieux l'algorithme est celui d'un troupeau d'animaux, où un ensemble des sous-actions de la part de chaque individu permettent une décision générale pour le troupeau ~\cite{CompGamesWithIntel}.

\section{Apprentissage supervisé et non-supervisé}
Comme décrit dans ~\cite{tambellinicomment}, deux méthodes d’apprentissages sont possibles : l’apprentissage supervisé et l’apprentissage non supervisé.
\subsection{Apprentissage supervisé}
Cette méthode demande une base d’exemple sur laquelle fonder l’apprentissage, en l’occurrence, l’enregistrement d’un joueur humain jouant au jeu. L’IA analyse la situation dans laquelle se trouve le joueur et enregistre ses actions, pour les reproduire lorsqu’elle se retrouve dans la même situation. Cette méthode est assez contraignante pour les développeurs qui doivent passer un certain temps à jouer pour rendre l’IA efficace.
\subsection{Apprentissage non-supervisé}
Cela consiste à créer des intelligences artificielles qui s’adaptent toutes seules à leur environnement. Cette méthode a l’avantage d’être assez légère pour le développeur qui doit juste créer un système de "récompense" quand l’IA agit correctement, et inversement quand elle fait des erreurs. Mais elle est plus difficile à réaliser, et plus souvent sujette à des erreurs non prévues. Si on prévoit mal la réaction de l’IA par rapport à ses stimulations extérieures, on risque de rentrer dans un comportement illogique.


\chapter{Systèmes multi-agents}
Le fait que les IA dans les jeux vidéos sont souvent associées à des personnages fait que celles-ci s'adapdent très bien au concept des agents en tant qu'IA, et plus particulièrement à un systèmes-multi agent dans le cas ou plusieurs personnages doivent coexister.
Ces agents sont généralement composés d'un ensemble de capteurs leur permettant d'obtenir un certain nombre d'informations sur leur environnement (objets, autres agents, joueurs, sons, etc.), sur eux-même (faim, vie, alignement, etc.), ainsi que d'un ensemble d'actions réalisables et de but ~\cite{IntelAgents4CompGames}. A partir ces informations, et en fonction de son implémentation, un agent déterminera un comportement à avoir.
Ces agents peuvent être classés en plusiseurs catégories.

\section{Les agents réactif}
Les agents réactifs sont la version la plus simple des agents, et sont très similaires à des machines à état fini précédemment décrites. Pour certaines informations obtenues par les capteurs, certaines actions seront effectuées. Il présente donc les mêmes avantages, c'est-à-dire qu'ils sont simples à mettre en place, et demande peu de puissance de calcul, et c'est pourquoi ils sont très utilisés.
Cependant, ils ont aussi les mêmes défauts, c'est-à-dire difficile à maintenir dans la cas d'un grand nombre de règles, et pouvant entrainer des actions erronées ou illogiques au regard du joueur.

\section{Les agents délibératifs}
Les agents délibératifs sont plus complexes que les réactifs, car au lieu d'exécuter une action pour certaines entrées, ceux-ci vont ``construire'' un plan à partir des actions qu'ils peuvent exécuter pour arriver à leur but ~\cite{IntelAgents4CompGames} ~\cite{IntelAgentsInCompGames}. Cela va conduire les agents à former une séquence d'actions afin de modifier le monde dans le sens qu'il souhaite.

Ainsi, il n'est pas nécessaire d'implémenter chaque règle comme avec un agent réactif, puisque à partir des ses connaissances et de son(ses) but(s), et connaissant ses actions possibles, va établir un plan afin d'arriver à son(ses) but(s).\\
Une implémentation connue de ce type d'agents est le modèle BDI (Belives - Desires - Intentions).\\
Le problème de ce type d'agents est que dans des cas complexes, ils peuvent demander un temps de calcul élevé car si au cours de la réalisation de son plan, des paramètres changent et font que celui-ci n'est plus réalisable, alors l'ensemble du plan doit être recalculé.

\section{Les agents hybrides}
Le but des agents hybrides, est de combiner la rapidité d'action d'un agent réactif, et l'intelligence (relative) d'un agent délibératif.
Pour cela, un temps de calcul fixe est donné à l'agent, et tant que celui-ci n'est pas dépassé, l'agent va améliorer et faire évoluer son plan afin de s'adapter aux changements de son environnement. Cependant, une fois les temps de calcul dépassé, l'agent va exécuter l'action suivante de son plan.

Cela permet de mettre en place un agent relativement intelligent, s'adaptant à son environnement comme un agent délibératif, mais ne calculant pas forcement l'ensemble du plan à chaque changement, et permettant une rapidité accrue.\\
Le seul paramètre reste alors le temps de calcul alloué à l'agent. Plus l'agent aura de temps, et plus son plan ira loin dans le futur et sera optimisé.

\section{Un exemple d'application}
En 2001, Lionhead Studios a sorti le jeu vidéo Black \& White. Celui-ci a été reconnu pour les qualités de son IA. En effet, les developpeurs du jeu ont voulu rendre cette partie la plus plausible et évolutive possible, et il en résulte une IA restant comme un exemple à suivre dans le monde du jeu vidéo aujourd'hui ~\cite{MindGames} ~\cite{CompGamesWithIntel}.\\
Dans ce jeu, le joueur prend la place d'un dieu dans un monde imaginaire, habité par différents peuples. Ce dieu est représenté dans le monde par une créature, dont l'IA évolue avec les actions du joueur.\\

L'ensemble des IA du jeu sont implémentée sous la forme de BDI, et ont la capacité d'apprendre de nouveaux buts, des priorités, des méthodes (ainsi que leur efficacité), etc. La créature représentant le joueur est autonome et apprend des actions de celui-ci et de l'observation des autres agents que faire, mais le joueur peut également faire évoluer l'IA lui-même par un système de récompenses / punitions. Les différents villageois du jeu sont eux controllés par un agent représentant un esprit de groupe et donnant aux villageois des désirs / buts. Ils agissent alors ensuite seuls.\\
Finalement, l'ensemble des ces méthodes permettent au jeu d'être vivant et surtout crédible, et c'est pour cela qu'il est un exemple.


\chapter{Modèlisation de la personnalité et des relation sociales}
Lorsque l'on veut intégrer dans un jeu vidéo des PNJ assez développés, intervient un moment ou l'on se heurte au problème de la personnalité du personnage. En effet si on veut que les différents PNJ n'aient pas les mêmes réactions face à ce qui leur arrive, il faut pouvoir leur donner une personnalité. Par exemple, un villageois et un chevalier ne réagiront pas de la même façon face à une attaque : on peut considérer que cela leur vient de la différence dans leur niveau de courage. De même, si on veut mettre en place une relation particulière vis-à-vis du joueur ou des autres personnages, il faut pouvoir représenter les émotions des personnages, et que celles-ci influent sur leurs réactions sociales. Par exemple, un personnage sera moins enclin à donner des informations au joueur si celui-ci l'a attaqué que s’il l'a aidé. 

\section{Modèles de personnalité}
D'après ~\cite{ochs2009simulation}, Différentes approches existent pour représenter la personnalité d'un personnage :
\begin{itemize}
\item L'approche par catégorie : on spécifie des comportements en fonction d'une catégorie comme par exemple « gentil/méchant » ou « peureux/courageux » comme dans le cas du villageois et du chevalier. Les différentes catégories sont spécifiées à l'avance dans le code du jeu et n'évoluent pas.
\item L'approche qui se base sur des traits de caractères. Par exemple, une capacité de « courage » peut être utilisée pour savoir si le personnage va répondre à une attaque en fuyant ou en se battant. Cette méthode permet par la suite d'implémenter une modification des attributs en cours de jeu, en fonction de ce qui arrive au personnage.  
\end{itemize}
Les représentations de personnalités sont souvent basées sur des modèles proposés par des grands noms de la psychologie comme ceux cités par ~\cite{ochs2009simulation} : Le modèle de personnalité de Hans Eysenck (qui considère 2 dimensions: Extroversion/ Introversion et le Névrosisme/Stabilité émotionnelle), Le modèle OCEAN (pour Ouverture à l'expérience, Conscienciosité, Extraversion, Agréabilité, et Névrosisme), Le MBTI (Myers Briggs Type Indicator) qui fonctionne avec 4 dimensions : Extraversion/Introversion, Sensation/Intuition, Pensée/Sentiments, Perception/Jugement ... \\
L'utilisation de ces modèles pour concevoir des IA de jeu vidéo permet d'améliorer leur crédibilité pour le joueur.

\section{Relations sociales}
Dans un jeu vidéo, les PNJ sont généralement là pour interagir avec le joueur, et, si on veut que leur comportement soit vraisemblable, il faut mettre en place une représentation des relations sociales entre les personnages. \\

Les relations sont généralement représentées, chez chaque personnage par un nombre fini de variables, qui s’appliquent à un autre personnage (le joueur ou bien un autre PNJ) avec qui il est déjà entré en communication. 4 variables sont généralement utilisées :
\begin{itemize}
\item Le degré d'appréciation que le personnage a pour un autre.
\item la dominance, c'est-à-dire le pouvoir que le personnage pense pouvoir exercer sur l'autre (et si elle est négative, c'est le pouvoir qu'il pense que l'autre peut exercer sur lui).
\item la solidarité, qui correspond au degré de similarité entre les contextes des 2 personnages, par exemple, l'âge, le sexe, les valeurs, les croyances ...
\item la familiarité, qui va caractériser la tendance du personnage à échanger des informations avec un autre.
\end{itemize}
Ces variables sont spécifiques à chacun des 2 personnages qui sont en relation et peuvent évoluer symétriquement ou non. Leur évolution se fait en fonction des interactions, par exemple, plus deux personnages parlent ensemble, plus leur niveau de familiarité augmente. Et réciproquement, ces variables influent sur les interactions possibles entre les personnages, par exemple, leur niveau de familiarité doit atteindre un certain seuil avant qu'ils ne puisse partager des informations confidentielles. 


\chapter{Conclusion}

\chapter{Bibliographie}
\begingroup
\def\chapter*#1{}
\bibliography{sources/bibliographieIA}{}
\bibliographystyle{plain}
\endgroup

\chapter{Liste des articles consultés}
\begin{thebibliography}{}
\bibitem{mahdi2013level},
  title={Level Of Detail Based AI Adaptation for Agents in Video Games},
  author={Mahdi, Ghulam and Francillette, Yannick and Gouaich, Abdelkader and Michel, Fabien and Hocine, Nadia and others},
  booktitle={ICAART'2013: 5th International Conference on Agents and Artificial Intelligence},
  year={2013}
}

\end{thebibliography}


\end{document}
